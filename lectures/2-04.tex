%!TEX root = ../notes.tex
\section{February 4th -- Imagining Community}

From last time: Only two states allow felons to vote, even from prison. All other states inhibit civil citizenship rights of felons. Thirteen states allow felons to vote after prison release.

Two women's stories similar, but with very different tones. Thinking about Truillot's silenced history -- what role does sexual shame play?

\subsection{Matoaka}

Also known as Pocohontas. c. 1595-1617. Married John Rolfe.

Folklore: falls madly in love with Captain John Smith. Loves him so much that she rescues him when he is about to be killed by her father. He returns to England because of an injury. Then she marries John Rolfe, gives birth to son, dies at 22.

Story told by John Smith, published in 1819 (long after his death). He started telling the story 15 years after it supposedly happened. Really, likely forced to married John Rolfe. Smith likely fabricated the aggression of the tribe to justify later actions.

Many of first families of Virginia trace their lineage to Pocohontas (people who identify as unequivocally white).

The Pocohontas Exception of the Racial Integrity Act of Virginia. In 1924, everyone put into a registry in which everyone was white or colored. White person cannot marry any non-white person unless they have $1\over16$ or less American Indian blood.

\subsection{Malinche}

Mother of \textit{La Raza}, Treacherous Chingada. Just after Spanish invaded modern Mexico, declared it new Spain. Knew Aztec language, a type of Mayan, and learned Spanish quickly. Spanish became dependant upon her for translation.

Pocohontas became `Rebecca' when she married John Rolfe. 

\paragraph{Repartimento} A colonial labor regime. Indigenous  Indigenous women given to spanish men as concubines. Sexual slavery runs through narratives told about her, but not Matoaka.

Had a son with Cortez, so in some ways she is held up as the mother of \textit{La Raza}, the ``Mexican race.'' Both the stories of these women come from other people's stories about them.

How to mythologies of so-called interracial sex influence shared histories? How are the narratives politicized? How do they reflect different people's access to power?

Malinche's nickname: \textit{La Chingada}: ``the screwed one.''

\paragraph{Imagined Communities} Differentiates between actual community and imagined community. The difference between an entryway and a nation. In entryway, we identify with people because we actually interact. Limits efficacy of interaction to shape shared interest. Why would people risk their lives for ``the nation?''

A certain familiarity to stores like Pocohontas that gets rooted in the minds of the nation.

\paragraph{Hernan Cortes} 1485-1547. Diaz wrote a story as a soldier under Cortes. Written when Diaz when was 80.
Malinche was baptized, took name Dona Marina. Malinchista means traitor, someone who sleeps with the enemy. This word has been encoded as a feminine word because of its association of a woman who supposedly betrayed her race/nation.

Thinking about how this story has become part of a nation's common sense (the protests when a statue of her and Cortes was put up). The opinion of her was more negative when Mexico was fighting for independence from Spain (until 1821). They became independent, only to find the the colossus of the north imposing upon Mexican sovereignty.

Celebratory image as the mother of the first ``Mestizo'' in 1925. The ambivalence and shame of her legacy live on -- 1950 ``The Labyrinth of Solitude'' discusses Mexican culture caught between indigenous peoples and European occupiers. Portrays Melinche as a victim. Chicana feminists in the 1970s tried to reclaim the maligned figure, emphasized her as a powerful, intelligent woman who navigates a complex path to survival at a time of violence.

On Wednesday, we continue looking at stories of so-called interracial sex. Note that there are many different definitions of interracial sex -- up until 1967, states had different definitions of this term. Also, note that these laws are only to prevent anyone white from marrying someone non-white. Think about why.

Historian Gary Nash asks why the United States didn't become a ``mestizo nation.''
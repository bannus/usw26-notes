%!TEX root = ../notes.tex
\section{February 6th -- Melting Pots, Metissage, and Mestizo Origin Stories}

Thinking about braided citizenship -- a braid is a set of ropes intertwined in a non-linear way. Also, brings different elements together in unexpected ways.

Colonial legacies of the braided citizenship. Effects of race, gender, sex, class.

Truillot - ``every history is a particular bundle of silences.'' Silences at four moments of historical production. ``Power begins at the source.'' Fact creation (sources), fact assembly (archive), fact retrieval (narrative), retrospective significance (history).

\paragraph{Fact Creation} Thinking about Matoaka/Malinche, what narratives do we have? Third party stories, their lineage. Why do we lack narratives directly from them? At least in the case of Malinche, we know she was very literate. Thinking about the balance of power at this time -- thought that Matoaka was enslaved at some point. Also some thought that they resisted the system, tried to navigate the system to take power for themselves.

\paragraph{Fact Assembly} Human intentionality -- the choices that go into deciding what to save

\paragraph{Fact Retrieval} The stories that were told by others after the fact. Not only subject to the silences of the historical generation already happened, but also the fallibility of human memory, as well as the perceived importance of the person telling the story.

\paragraph{Retrospective significance} Significant stories speak volumes about the aspirations and thoughts of the nation.

\paragraph{Captivity Narratives} First `best-seller' of the colonies. Ambivalent portrayal of Native Americans -- she befriended them. Also described them as savages because of their religious differences. Material consequences from a book like this -- idea that Native American men sought to capture/rape women (even though rape never mentioned), justifying later violence against Native Americans.

\paragraph{Racial formation} ``The sociohistorical process by which racial categories are created, inhabited, transformed, and destroyed.'' Race is a power-infused process, not a natural or biological fact. 

\paragraph{Castas paintings} Spanish, 17th/18th c. new Spain (Mexico). Emerged as a way of depicting class differentials between people. Today, we view it as a formative part of our modern idea of race. 

Harvard had an Indian College est. 1655

Castas paintings depict various marital pairings and offspring. Idea of `one drop rule' already present in the blackness of some offspring.
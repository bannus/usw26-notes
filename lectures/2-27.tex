%!TEX root = ../notes.tex
\section{February 27th -- The Politics of Exclusion} 

U.S. Census excluded Native people until 1860.

\begin{itemize}
  \item 1862 Homestead act: denied to those who had raised arms against the country (Civil War)
\end{itemize}

``An Amalgamation Waltz'' -- 1839 propaganda based on notions of interracial mixing. ``Practical Amalgamation (The Dinner Party)'' -- also 1839. 

How do these `texts' help us understand racial formation? What kind of affect would these images have generated among their audience (white citizens).

Sense of who stands to gain from racial equality, who will lose. Sense of hostility from white audience. Even among abolitionists, fear to take it as far as racial equality, because of implications of interracial sex (especially black men/white women, as depicted here).

Crowley text from 1860s `If you eliminate slavery, this will happen' -- attempt to `embarrass' Republican party. 

Marriage is a way that people are treated as part of the `family' of the state. Laws decided who you cannot marry define citizenship/belonging.

Different states had a different ideas about interracial marriage. Laws calling something `interracial' actively constructs racial boundaries. Changes way citizens view each other. Did not outlaw marriage between racially minoritized people.

\paragraph{Power of Science} Science is influencing the kind of laws the state develops. In 19th c., science becomes supreme mode of categorizing/understanding human body. Belief in the heritability of certain characteristics (blood, interracial sex produces impure blood). Even though Darwin proposes that all humans are descended from a single ancestor, thought that the existence of races represent different species. Production of science is inextricable from human subjectivity -- the biases of the day. Late 19th c. growth of cities, industrialization -- science becomes a `solution' that will solve the problems of overpopulation (`Liberatory Biologism').

Physiognomy -- the belief that deviant/pathologized body has visible evidence of immorality contained within. Immigration officials charged with allowing citizens/excluding. LPC category -- likely to become a public charge. Belief that trained officials could determine that an immigrant would become a vagrant. Women and children traveling alone without a male protector are singled out.

\paragraph{The Page Law} 1875, First U.S. immigration law excluding undesirable categories of human beings. Sound like it is protecting people from forced labor jobs, also protecting women from prostitution? But what it actually does is criminalize/exclude certain types of people -- targeted at Chinese. Precursor to 1882 law explicitly forbidding non-wealthy Chinese. Far more men coming in from China.

`Yellow Peril' -- notion that Chinese immigrants were unassimilable. Similar gesture to the Amalgamation Waltz pictures -- disenfranchisement of whites.'

\paragraph{14th Amendment} -- equal protection clause. Shift from jus sanguinis (blood right) to jus soli (born in U.S.). Expands definition of citizenship. Designed to guarantee former slaves citizenship.

\paragraph{15th Amendment} -- debate about whether to include sex. Extension of `political' citizenship to all \textit{men}.
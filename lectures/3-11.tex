%!TEX root = ../notes.tex
\section{March 11th -- }

\paragraph{Gender} Socially constructed roles, behaviors, activities, and attributes that a given society considers appropriate for men and women

\paragraph{Sex} Biological and physiological characteristics that define men and women (definitions from WHO)

19th c. -- shift in the organization of families. Previously site of production. All family members would make clothing, help farm (internally producing), shift to consumer-based model. Major repercussions for gender norms (husband becomes breadwinner working outside home). Feeds in to notion of ``separate spheres,'' with husband in public sphere, domestic realm considered feminine realm.

\paragraph{Cult of True Womanhood} 19th c. ideals of femininity. Shift in organization drives apart space occupied by men/women. Idea that women have superior piety -- women provide and anchoring point between church and home, help keep husbands in check. Purity/passionlessness -- no sex before marriage, and then only for procreation. Ideal woman thought to be passive, never argued, submits self to will of husband. Domesticity: private home provides protection for women/family from outside world.

Wealthy educated African American women were excluded from the Cult of True Womanhood. Belief that oriental women are predisposed to prostitution, excluded as well.

Charles McCord -- \textit{The American Negro...} Argument: all African Americans should be denied political citizenship rights. Based his argument on African Americans' deviant sexual behavior. Pulled on the ``scientific'' understanding of race, saying sexual purity rare among women, not seen among men.

Ida B. Wells -- witnessed several of her friends for being successful African Americans, in the name of ``protecting white womanhood.'' Bold response specifically targets mythology of sexuality in 1892. Circulates this pamphlet in Europe, calls into question the civilization of white America, telling the world about an uncivilized vigilante justice that dominates.

\paragraph{White Slavery} The idea that immigrant men were responsible for trafficking white women. Based on idea that white women who were prostitutes were forced into it, since they do not seek sex. Certain groups of European immigrants thought to be racially suspect (race getting fragmented).

Anxiety about white women's natural delicacy. Hysteria -- ancient that women's wombs can wander their bodies. Melancholia/weakness/anxiety treated by stimulation to orgasm by male doctors. Not thought to be sexual -- a medical procedure. Extreme measures to inhibit reproduction of ``perverts''/``inverts'', culminating in forced sterilization in early 20th century.

Couverture still alive, belief that women could not have independent political beliefs, relied on men for protection. This idea shows up in naturalization policy. 1855: any woman can gain citizenship by marrying a US citizen (marriage trumps existing law). Notably not up to the woman -- no application needed. Only way Chinese descended people to become citizens is to be born in US. 1907: female U.S. citizens lose citizenship if married alien man. Disincentive for any citizen to marry alien (notably: Chinese) men.

\paragraph{The Invert} Invented in 1860s, more in keeping with our current understanding of homosexuality. National scandal: Alice Mitchell murdered her lover Freda Ward. Spotlight on inversion. Alice became jealous when Freda said she would marry a man. Havelock Ellis uses Alice Mitchell as a case for understanding the female invert. He finds history of insanity in her family -- sees something congenital about her violence and inversion. Also, her asymmetrical face gave her away (physiognomy), and that she ``affected a certain degree of masculinity.'' Unclear: did she wear men's clothes? Did she assume male gender roles?

Anxiety producing/enigmatic in late 19th/20th century. Women and men very sex-segregated, women expected to exhibit certain amount of affection toward other women. Female inverts only visible when the dressed up and ``acted the part'' of the invert.

\paragraph{Florida Enchantment} 1914 silent film. Magic seeds invert gender. The woman (Lillian) does fine in her transformation, while Fred looks awkward and is chased by angry mob into a lake. Fear of female inversion pervades culture at the time, racialized concern. Neurasthenia has a male counterpart. Theodore Roosevelt is concerned about this, encourages men to take part in masculine activity. Represents more than a challenged to male supremacy, also idea that modernity has affected men.

\paragraph{Theodore Roosevelt} Starts as a sickly young man, took pride in wrestling/hunting, transformed image. Gets deeply engaged in efforts to recover failed white masculinity. Also coins term ``race suicide.''

Broken Blossoms -- American soldiers fighting, playful, undisciplined. Provide explanation for protagonist to go to Europe. Seemingly celebratory image of Chinese undermines their image
%!TEX root = ../notes.tex
\section{March 25th -- Loving and Precurorsors: ``Unnatural'' Sex in the Early 1900s}

Harvard ``Secret Court'' -- secret interviews conducted by tribunal of administrators. 8 students disciplined for ``unnatural acts.''

\paragraph{Hegemony} Looking at how the ruling classes retain power without using force. Antonio Gramsci (1891-1937) looked at how ruling classes were able to project their ideas of ``natural'' and ``common sense.'' Emphasized that hegemony is a process, never stagnant.

What silences are missing if we look at court cases? They are a good example of people standing up to hegemony. People who were afraid of/lacked access to courts.

\paragraph{Exclusion and Sterilization} State efforts to regulate citizenry. Laws based on common sense understanding (hegemonic?) of how to protect population.

Why did so few people fight back against the idea of ``whiteness'' upon which miscegenation/immigration laws were based?

Even European immigrants conisdered ``suspect'' were considered to be assimiliable, and were counted in immigration quotas. The courts decided the white, legitimate, from the non-white, non-legitimate.

\paragraph{Race vs. Ethnicity} Ethnicity was not used in early 20th c. -- emerged in mid. In early 20th c., people talked about race, but used term in different ways. Race -- immutable versus ethnicity -- transformable. Ethnicity: traditional and cultural origins. Race: rooted in biological origins.

Mae Ngai -- least desirable forms of immigrants (as determined by Dillingham commission) thought to be capable of assimilation to U.S. norms. Even those from southern/eastern Europe would assimilate ethnically and also acquire white racial identity.

``Mongolians'' -- thought to be racially unassimilable.

1875 Page Act -- targets Asian women (first instance of racial excludability), Asian workers.  1878 -- notion of Caucasian becomes a legal category. People of ``Mongolian race'' would not count as ``white'' because they were not Caucasian.

Ngai -- Lawmakers invoked anthropology, scientific racism. Race and nationality in 19th c. were conflated. By early 1900s, they are dissociated. 

A quota that reflects the way policy-makers want the nation to look. 1924 National Origins Law.


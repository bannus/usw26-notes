%!TEX root = ../notes.tex
\section{April 8th -- The Politics of Passing (\& Covering): Being and Becoming \textit{American}}

Thinking about how people transition into ``ideal'' citizens. Continuing the story of the transition of Asian Americans into ``model minority.''

Slide: various steps in relaxation of racial exclusion laws. The state's investment in treating people differently according to race is starting to erode. State's number one priority: familialism. Celebration of the heterosexual family. 

\paragraph{McCarran-Walter} Actually heightens state's ability to deport people at the start of the Red Scare. 1930s -- effort to stop KKK by House on American Activities, turns to anti-communist agency. There was an idea that communist affiliation of any kind was fundamentally treasonous. McCarthy -- campaign to weed out the enemy within. At one point, 69\% believed he was doing a good job of protecting the nation -- widespread fear.


\paragraph{State homophobia} State effort to expunge homosexuals from the military and government. Why did this arise? Emphasis to American family as basic unit of society, and homosexuality is seen as a challenge to that. Thought that state's engagement in patrolling behaviour in private settings is justified in the name of protecting the nation. Homosexuals in government seen as a security threat, as they could be blackmailed and expose secrets to protect themselves.

\paragraph{Post WWII Red Scare} Julius and Ethel Rosenberg executed in 1953 for selling secrets to the Russians. Later determines that Ethel was uninvolved. Familialism celebrated in Baby Boom -- fashions, TV shows, advertising copy celebrate the feminine woman, masculine man. Suspicion of people who don't perform their gender correctly.

How do Asian Americans become model minority by 1960s -- why doesn't similar thing happen to Hispanics/Blacks?

\paragraph{Passing and Covering} Sally Hemings/Clotel are light skinned, are able to ``pass'' as white (Clotel pretends to be a man as well). Causing others to believe that you're something that you're not. Usually becoming the majority to enjoy privileges/acceptance. Eddie Murphy passing as white in a parody of book ``Black Like Me'' -- Mulan passing as a man. Requires more effort for differences etched on the body.

Differentiating passing from what Yoshino (sp?) describes as ``covering.'' Passing denotes a calculated effort to deceieve one's audience (the people around you). Covering is more subtle -- involves downplaying minoritized differences. Doesn't necessary deny affiliation, but downplays characteristics that differentiate themselves from the norm.

Monolithic category of ``Asian-ness'', different categories brought back together in 1960s, collectively stereotyped. Covering is harder to trace -- have to determine authentic self of an individual. What kinds of compulsions prevail in our cultural surround?

\paragraph{Cultural Citizenship} One downside of seemingly positive stereotypes -- expectations based on a visual assessment. What happens when someone characterized as a certain kind of model minority acts outside of the associated stereotypes?

Blackface becomes more shameful in the 1950s, as a shift from hard racialism. 1920s - Al Jolson was a Jewish man, most famous blackface performer.
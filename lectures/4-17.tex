%!TEX root = ../notes.tex
\section{April 17th -- Policing the Boundaries of the Home}

Guest lecture by Claire Houston.

The public/private distinction, through the lens of policing the boundaries of the home. Definition of these terms is constantly shifting due to politics.

\paragraph{Public/private distinction} Many interpretations. State/civil society. Market/family. Two meanings can work together. The idea that there could be a private space is a relatively new idea. Influenced by period of nation states. Parliaments have unconstrained power to make law, people made effort to stake out private spheres outside influence of state. Family/market did not come about until 19th c. when market became a major force.

Focus will be on private sphere -- will draw on both meanings of private. Home as locus of ``effective lives.'' Where the family lives.

State regulation of market can combat rampant capitalism, regulation of family is seen as totalitarian. One sphere (market) becomes more public, other (family) becomes more private.

\paragraph{McGuire v. McGuire} Heard by Supreme Court of Nebraska. Married couple. Lydia suing Charles for not providing enough financial support for the home.

\begin{itemize}
	\item Issue: Is Lydia entitled to support even though they are living together?
	\item Ratio (rule): Support can be awarded to a wife who continues to be married. (Cited cases are of women/men not living together).
	\item Analysis: Majority does not believe they should extend the rule. State should not be intervening, as long as parties are together, should assume that support is adequate.
	\item Conclusion: Assumes there is a sphere which the state is not permitted to enter. Can be interpreted as affirming inequality between husbands and wives. 
\end{itemize}

\paragraph{Griswold v. Connecticut} This case often read alongside Griswold v. Connecticut (1965). Invalidated CT law that prohibited use of birth control. Notion that there is a space in which government should not interfere, trying to extend this to the media. Not arguing for a constitutional right to privacy -- constitution only protects people against state acts.

Language used reaffirms both meanings of public/private. Emphasizes sacred/intimate nature of the home, distinct from other spheres that could be subject to government intervention (such as market).

Loving v. Virginia challenged law, determined to be violation of equal protection clause. Procedural due process clause protects you from state deprivation without due process of law. Substantive due process -- certain rights that have always existed, not enumerated in constitution, that state can't take away. This is where privacy comes from.

Next case after Griswold to recognize right to privacy Roe v. Wade, provides right to make a decision whether or not to terminate a pregnancy. Had to be balanced against state's interest to protect fetal life and women's health. Implicit that women's decision is not privately her own (doctors involved). Simultaneously, homosexuality removed from mental disorders.

\paragraph{Lawrence v. Texas} Return to the private home. Conduct that can occur in some sphere without state intervention affirms state/civil society. Distinction between private sex, of no interest to state, and private sex (prostitution) which occurs in the context of the market.

\paragraph{Domestic Violence} Marriage: Ceding of body/property to husband. Wife's legal identity merged into that of her husband, had to access legal system through him. Right of chastisement: anything short of personal injury allowed by husbands against wife, children, servants. By 1920s, states had repudiated chastisement doctrine. Before, legally state was restricted from intervening in home, afterwards, state kept out by public/private distinction.

\paragraph{DeShaney (1989)} Suit against Department of Social Services for failure to provide protection from abusive father. Ruling says DSS was smart to err on side of parents.

